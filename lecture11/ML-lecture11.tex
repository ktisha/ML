\documentclass[12pt]{beamer}
\usepackage{cmap}
\usepackage[T2A]{fontenc}
\usepackage[utf8]{inputenc}
\usepackage{ifluatex}
\usefonttheme[onlymath]{serif}
\usepackage{svg}
\usepackage{enumerate}
\usepackage{hyperref}
\usepackage{mathtools}
\setbeamertemplate{footline}[frame number]
\definecolor{beamer@darkgreen}{rgb}{0,0.6,0}
\setbeamercolor{normal text}{fg=black,bg=white}
\setbeamercolor{title}{fg=black,bg=beamer@darkgreen}
\setbeamercolor{frametitle}{fg=black,bg=beamer@darkgreen}
\setbeamercolor{background canvas}{parent=normal text}

\usepackage[english,russian]{babel}
\usepackage{graphicx}
\usepackage{listings}
\DeclareMathOperator{\sign}{sign}

\usepackage{enumerate}

\author{Катя Тузова}
\title{Машинное обучение}
\date{}

\usepackage{gensymb}

\subtitle{Лекция 11. Линейная регрессия. PCA. AUC.}

\begin{document}	
\frame{\titlepage}

\begin{frame}\frametitle{Разбор летучки}

\end{frame}

\begin{frame}\frametitle{Метод наименьших квадратов}
В первом домашнем задании мы реализовывали метод наименьших квадратов.\\
К какому типу классификаторов он относится?
\end{frame}

\begin{frame}\frametitle{Регрессия}
$X$-- объекты в $\mathbb{R}^n$; Y — ответы в $\mathbb{R}$\\
$X^l = (x_i, y_i)_{i=1}^l$ -- обучающая выборка\\
$y_i = y(x_i)$,  $y : X \rightarrow Y$ -- неизвестная зависимость\\
\vspace{5mm}
$a(x) = f (x, \alpha)$ -- модель зависимости,\\
$\alpha \in \mathbb{R}^p$ -- вектор параметров модели.\\
\vspace{5mm}
Метод наименьших квадратов (МНК):\\
$Q(\alpha,X^l) = \sum\limits_{i=1}^l w_i (f (x_i, \alpha) - y_i)^2 \rightarrow \min\limits_{\alpha}$\\
где $w_i$ -- вес, степень важности i-го объекта.\\
$Q(\alpha^*,X^l)$ — остаточная сумма квадратов

\end{frame}

\begin{frame}\frametitle{Метод максимума правдоподобия}
Модель данных с некоррелированным гауссовским шумом:\\
$y(x_i) = f (x_i, \alpha) + \varepsilon_i$, $\varepsilon_i \in N (0, \sigma_i^2), i = 1, \dots, l$\\
Метод максимума правдоподобия (ММП):\\
$L(\varepsilon_1, \dots, \varepsilon_l | \alpha) = \prod\limits_{i=1}^l \frac{1}{\sigma_i \sqrt{2\pi}} \exp (-\frac{1}{2\sigma_i^2} \varepsilon_i^2 ) \rightarrow \max\limits_{\alpha}$\\
$- \ln L(\varepsilon_1, \dots, \varepsilon_l| \alpha) = const(\alpha) + \frac{1}{2} \sum\limits_{i=1}^l \frac{1}{\sigma_i^2} (f(x_i, \alpha) - y_i)^2 \rightarrow \min\limits_{\alpha}$\\
\vspace{5mm}
Постановки МНК и ММП, совпадают, причём веса объектов
обратно пропорциональны дисперсии шума, $w_i = \sigma_i^{-2}$
\end{frame}

\begin{frame}\frametitle{Многомерная линейная регрессия}
$f_1(x), \dots, f_n(x)$ -- числовые признаки\\
Модель многомерной линейной регрессии:\\
$$f (x, \alpha) = \sum\limits_{i=1}^l \alpha_j f_j(x), \alpha \in \mathbb{R}$$\\
\vspace{5mm}
Функционал квадрата ошибки:\\
$$Q(\alpha,X^l) = \sum\limits_{i=1}^l (f (x_i, \alpha) - y_i)^2  = \Vert F\alpha - y \Vert^2 \rightarrow \min\limits_{\alpha}$$\\

\end{frame}

\begin{frame}\frametitle{Нормальная система уравнений}
Необходимое условие минимума в матричном виде:\\
$$\frac{\partial Q}{\partial \alpha} (\alpha) = 2 F^T(F \alpha - y) = 0$$\\
Откуда следует нормальная система задачи МНК:\\
$$F^TF\alpha = F^Ty$$\\
$F^TF$ -- ковариационная матрица набора признаков $f_1, \dots, f_n$\\
\vspace{5mm}
Решение системы:\\
$$\alpha^* = (F^TF)^{-1}F^Ty = F^+y$$\\
Значение функционала: $Q(\alpha^*) = \Vert P_F y - y \Vert^2$,\\
где $P_F = FF^+ = F(F^TF)^{-1}F^T$ -- проекционная матрица
\end{frame}

\begin{frame}\frametitle{Сингулярное разложение}
Произвольная $l \times n$-матрица представима в виде сингулярного разложения:\\
$F = VDU^T$\\
Основные свойства сингулярного разложения:\\
\begin{enumerate}[--]
\item $l \times n$-матрица $V = (v_1, \dots, v_n)$ ортогональна, $V^TV = I_n$, столбцы $v_j$ - собственные векторы матрицы $FF$
\item $n \times n$-матрица $U = (u_1, \dots, u_n)$ ортогональна, $U^TU = I_n$, столбцы $u_j$ - собственные векторы матрицы $F^TF$
\item $n \times n$-матрица $D$ диагональна, $D = diag (\sqrt{\lambda_1}, \dots, \sqrt{\lambda_n})$, $\lambda_j > 0$ - собственные значения матриц $F^TF$ и $FF^T$
\end{enumerate}
\end{frame}

\begin{frame}\frametitle{Решение МНК через сингулярное разложение}
Псевдообратная $F^+$, вектор МНК-решения $\alpha^*$,
МНК-аппроксимация целевого вектора $F\alpha^*$\\
$F^+ = (UDV^TVDU^T)UDV^T = UD^{-1}V^T = \sum\limits_{j=1}^n \frac{1}{\sqrt{\lambda_j}}  u_j v_j^T$\\
$ \alpha^* = F^+y = UD^{-1}V^Ty = \sum\limits_{j=1}^n \frac{1}{\sqrt{\lambda_j}}  u_j (v_j^Ty)$\\
$F \alpha^* = P_F y = (VDU^T)UD^{-1}V^Ty = VV^Ty = \sum\limits_{j=1}^n v_j (v_j^Ty)$\\
$\Vert \alpha^* \Vert^2  = \Vert D^{-1}V^Ty \Vert^2 = \sum\limits_{j=1}^n \frac{1}{\sqrt{\lambda_j}} (v_j^Ty)^2$
\end{frame}

\begin{frame}\frametitle{Проблема мультиколлинеарности}
Если $\exists \gamma \in \mathbb{R}^n: F\gamma \approx 0$, то некоторые $\lambda_j$ близки к нулю.\\
Число обусловленности $n \times n$-матрицы $\Sigma$:
$$\mu(\Sigma) = \Vert \Sigma \Vert \Vert \Sigma^{-1} \Vert = \frac{\max\limits_{u: \Vert u \Vert = 1} \Vert \Sigma u \Vert}{\min\limits_{u: \Vert u \Vert = 1} \Vert \Sigma u \Vert} = \frac{\lambda_{max}}{\lambda_{min}}$$\\
При умножении обратной матрицы на вектор, $z = \Sigma^{-1}u$, относительная погрешность усиливается в $\mu(\Sigma)$ раз:\\
$$\frac{\Vert \delta z \Vert}{\Vert z \Vert } \leq \mu(\Sigma) \frac{\Vert \delta u \Vert}{\Vert u \Vert }$$
\end{frame}

\begin{frame}\frametitle{Проблема мультиколлинеарности}
Если матрица $\Sigma$ плохо обусловлена, то: % кто такая сигма
\begin{enumerate}[--]
\item решение становится неустойчивым и неинтерпретируемым, $\Vert \alpha^* \Vert $ велико
\item на обучении $Q(\alpha^*, X^l) = \Vert F\alpha^* -y \Vert$ -- мало   % и что из этого
\item на контроле $Q(\alpha^*, X^k) = \Vert F'\alpha^* -y' \Vert$ -- велико
\end{enumerate}
\end{frame}

\begin{frame}\frametitle{Проблема мультиколлинеарности}
Стратегии устранения мультиколлинеарности и переобучения:
\begin{enumerate}[--]
\item Отбор признаков
\item Регуляризация: $\Vert \alpha \Vert \rightarrow \min$
\item Преобразование признаков
\end{enumerate}
\end{frame}


\begin{frame}\frametitle{Гребневая регрессия}
Штраф за увеличение нормы вектора весов $\Vert \alpha \Vert$\\
$$Q_{\tau} (\alpha) = \Vert F \alpha - y \Vert^2 + \frac{1}{\sigma} \Vert \alpha \Vert^2$$\\
где $\tau = \frac{1}{\sigma}$ -- неотрицательный параметр регуляризации.\\
Вероятностная интерпретация: \\
Априорное распределение вектора $\alpha$ -- гауссовское с ковариационной матрицей $\sigma I_n$.\\
Модифицированное МНК-решение ($\tau I_n$ — «гребень»)\\
$$\alpha^*_{\tau} = (F^TF + \tau I_n)^{-1}F^Ty$$\\
Преимущество сингулярного разложения:\\
можно подбирать параметр $\tau$ , вычислив SVD только один раз.
\end{frame}

\begin{frame}\frametitle{Регуляризованный МНК через сингулярное разложение}
Вектор регуляризованного МНК-решения $\alpha_{tau}^*$ и МНК-аппроксимация целевого вектора $F\alpha_{\tau}^*$:\\
$$\alpha_{tau}^*= U(D^2 + \tau I_n)^{-1}DV^Ty = \sum\limits_{j=1}^n \frac{\sqrt{\lambda_j}}{\lambda_j + \tau} u_j(v_j^Ty)$$\\
$$F\alpha_{\tau}^* = VDU^T\alpha_{\tau}^* = V diag (\frac{\lambda_j}{\lambda_j + \tau}) V^Ty = \sum\limits_{j=1}^n \frac{\sqrt{\lambda_j}}{\lambda_j + \tau} v_j (v_j^Ty)$$\\
$$\Vert \alpha_{tau}^* \Vert^2 = \Vert (D^2 + \tau I_n)^{-1} DV^T y \Vert^2 \sum\limits_{j=1}^n \frac{\sqrt{\lambda_j}}{\lambda_j + \tau} (v^T_j y)^2$$\\
$F \alpha_{\tau}^* \neq F\alpha^*$ -- зато решение становится более устойчивым
\end{frame}

\begin{frame}\frametitle{Выбор параметра регуляризации $\tau$}
Контрольная выборка: $X^k = (x'_i, y)_{i=1}^k$\\
Вычисление функционала $Q$ на контрольных данных $T$ раз потребует $O(kn^2 + knT)$ операций:\\
$$Q(\alpha_{\tau}^*,X^k) = \Vert F' \alpha_{\tau}^* - y' \Vert^2 = \Vert F'U diag (\frac{\lambda_j}{\lambda_j + \tau}) V^T y -y' \Vert^2$$\\
Зависимость $Q(\tau)$ обычно имеет характерный минимум.
\end{frame}

\begin{frame}\frametitle{Регуляризация сокращает «эффективную размерность»}
Сокращение весов:\\
$$\Vert \alpha_{tau}^* \Vert^2 = \sum\limits_{j=1}^n \frac{\sqrt{\lambda_j}}{\lambda_j + \tau} (v^T_j y)^2 < \Vert \alpha^* \Vert^2  = \sum\limits_{j=1}^n \frac{1}{\sqrt{\lambda_j}} (v_j^Ty)^2$$\\
Роль размерности играет след проекционной матрицы:\\
$$tr F(F^TF)^{-1}F^T = tr(F^TF)^{-1}F^TF = tr I_n = n$$\\
При использовании регуляризации:\\
$$tr F(F^TF + \tau I_n)^{-1}F^T = tr \hspace{1mm}  diag(\frac{\sqrt{\lambda_j}}{\lambda_j + \tau})= \sum\limits_{j=1}^n \frac{\sqrt{\lambda_j}}{\lambda_j + \tau } < n $$
\end{frame}

\begin{frame}\frametitle{LASSO}
LASSO -- Least Absolute Shrinkage and Selection Operator\\
$$
\begin{cases} Q(\alpha,X^l) = \Vert F\alpha - y \Vert^2 \rightarrow \min\limits_{\alpha} \\
\sum_{j=1}^n \vert \alpha_j \vert \leq \kappa
\end{cases}
$$\\
После замены переменных:\\
$$
\begin{cases} \alpha_j = \alpha_j^+ - \alpha_j^-\\
\vert \alpha_j \vert = \alpha_j^+ + \alpha_j^-\\
\alpha_j^+, \alpha_j^- \geq 0\\
\end{cases}
$$\\
ограничения принимают канонический вид:\\
$$\sum\limits_{j=1}^n \alpha_j^+ + \alpha_j^- \leq \kappa$$\\
Чем меньше $\kappa$, тем больше $j$ таких, что $\alpha_j^+ = \alpha_j^- = 0$
\end{frame}

\begin{frame}\frametitle{Негладкие регуляризаторы}
Elastic Net:\\
$$\frac{1}{2} \Vert F\alpha - y \Vert^2 + \mu \sum\limits_{j=1}^n \vert \alpha_j \vert + \frac{\tau}{2} \sum\limits_{j=1}^n \rightarrow \min\limits_{\alpha}$$\\
Support Features Machine:\\
$$\frac{1}{2} \Vert F\alpha - y \Vert^2 + \tau \sum\limits_{j=1}^n R_{mu}(\alpha_j) \rightarrow \min\limits_{\alpha}$$\\
$$ R_{\mu}(\alpha_j) = \begin{cases} 2 \mu \vert \alpha_j \vert, \hspace{2mm} \vert \alpha_j \vert \leq \mu\\
\mu^2 + \alpha_j^2, \hspace{2mm} \vert \alpha_j \vert \geq \mu
\end{cases}$$\\
Применение этих методов требует выбора траектории регуляризации в пространстве $(\mu, \tau )$
\end{frame}

\begin{frame}\frametitle{Метод главных компонент}
$f_1(x), \dots, f_n(x)$ -- исходные числовые признаки\\
$g_1(x), \dots, g_m(x)$ -- новые числовые признаки, $m \times n$\\
Как сформулировать требование к новым признакам?
\end{frame}


\begin{frame}\frametitle{Метод главных компонент}
$f_1(x), \dots, f_n(x)$ -- исходные числовые признаки\\
$g_1(x), \dots, g_m(x)$ -- новые числовые признаки, $m \times n$\\
Требование: старые признаки должны линейно
восстанавливаться по новым:\\
$$\hat{f}_j(x) = \sum\limits_{s=1}^m g_s(x) u_{js} , \hspace{2mm} j = 1,\dots , n, \hspace{2mm} \forall x \in X$$\\
как можно точнее на обучающей выборке $x_1, \dots, x_l$:\\
$$\sum\limits_{i=1}^l \sum\limits_{j=1}^n (\hat{f}_j(x_i) - f_j(x_i))^2 \rightarrow \min\limits_{g_s(x_i), u_{js}}$$
\end{frame}

\begin{frame}\frametitle{Матричные обозначения}
$U$ -- матрица линейного преобразования новых признаков в старые\\
$$\hat{F} = GU^T \approx F$$\\
Найти: и новые признаки $G$, и преобразование $U$:\\
$$\sum\limits_{i=1}^l \sum\limits_{j=1}^n (\hat{f}_j(x_i) - f_j(x_i))^2 = \Vert GU^T - F \Vert^2 \rightarrow \min\limits_{G, U}$$
\end{frame}

\begin{frame}\frametitle{Основная теорема}
Если $m < rk F$, то минимум $\Vert GU^T - F \Vert^2$ достигается, когда столбцы $U$ - это с.в. матрицы $F^TF$, соответствующие $m$ максимальным с.з. $\lambda_1,\dots, \lambda_m$, а матрица $G = FU$.\\
При этом:\\
\begin{enumerate}[--]
\item матрица $U$ ортонормирована: $U^TU = I_m$
\item матрица $G$ ортогональна: $G^TG = \Lambda = diag(\lambda_1, \dots, \lambda_m)$
\item $U\Lambda = F^TFU$,  $G\Lambda = FF^TG$
\item $\Vert GU^T - F \Vert^2 = \Vert F \Vert^2 - tr \Lambda = \sum\limits_{j=m+1}^n \lambda_j$
\end{enumerate}

\end{frame}

\begin{frame}\frametitle{Связь с сингулярным разложением}
Если взять $m = n$, то:\\
\begin{enumerate}[--]
\item $\Vert GU^T - F \Vert^2 = 0$
\item представление $\hat{F} = GU^T = F$ точное и совпадает с сингулярным разложением при $G = V \sqrt{\Lambda}$
$$F = GU^T = V\sqrt{\Lambda}U^T, \hspace{2mm} U^TU = I_m, \hspace{2mm} V^TV = I_m$$
\item линейное преобразование $U$ работает в обе стороны:\\
$$F= GU^T, \hspace{4mm} G=FU$$
Поскольку новые признаки некоррелированы ($G^TG = \Lambda$), преобразование $U$ называется декоррелирующим
\end{enumerate}
\end{frame}

\begin{frame}\frametitle{Эффективная размерность выборки}
Упорядочим с.з. $F^TF$ по убыванию: $\lambda_1 > \dots > \lambda_n > 0$\\
Эффективная размерность выборки -- это наименьшее целое $m$, при котором\\
$$E_m = \frac{\Vert GU^T - F \Vert^2}{\Vert F \Vert^2} = \frac{\lambda_{m+1} + \dots + \lambda_{n}}{\lambda_1 + \dots + \lambda_n} \leq \varepsilon$$\\
Критерий «крутого склона»: находим $m: E_m-1 >> E_m$:
TODO:картинка
\end{frame}

\begin{frame}\frametitle{Решение задачи НК в новых признаках}
Заменим $F$ на её приближение $GU^T$:\\
$$\Vert GU^T\alpha -y \Vert^2 = \Vert G\beta -y \Vert^2 \rightarrow \min\limits_{\beta}$$\\
Связь нового и старого вектора коэффициентов:\\
$$\alpha = U\beta, \hspace{5mm} \beta = U^T \alpha$$\\
Решение задачи наименьших квадратов относительно $\beta$ (единственное отличие -- $m$ слагаемых вместо $n$):\\
$$\beta^* = D^{-1}V^Ty = \sum\limits_{j=1}^m \frac{1}{\sqrt{\lambda_j}} u_j (v_j^Ty)$$\\
$$G\beta^* = VV^Ty = \sum\limits_{j=1}^m v_j (v_j^Ty)$$
\end{frame}

\begin{frame}\frametitle{ROC кривая}
\end{frame}


\begin{frame}\frametitle{На следующей лекции}
\begin{itemize}
\item[--] Нейронные сети
\end{itemize}
\end{frame}
\end{document}
