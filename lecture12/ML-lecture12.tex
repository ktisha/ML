\documentclass[12pt]{beamer}
\usepackage{cmap}
\usepackage[T2A]{fontenc}
\usepackage[utf8]{inputenc}
\usepackage{ifluatex}
\usefonttheme[onlymath]{serif}
\usepackage{svg}
\usepackage{enumerate}
\usepackage{hyperref}
\usepackage{mathtools}
\setbeamertemplate{footline}[frame number]
\definecolor{beamer@darkgreen}{rgb}{0,0.6,0}
\setbeamercolor{normal text}{fg=black,bg=white}
\setbeamercolor{title}{fg=black,bg=beamer@darkgreen}
\setbeamercolor{frametitle}{fg=black,bg=beamer@darkgreen}
\setbeamercolor{background canvas}{parent=normal text}

\usepackage[english,russian]{babel}
\usepackage{graphicx}
\usepackage{listings}
\DeclareMathOperator{\sign}{sign}

\author{Катя Тузова}
\title{Машинное обучение}
\date{}

\usepackage{gensymb}

\subtitle{Лекция 11. Нейронные сети.}

\begin{document}	
\frame{\titlepage}

\begin{frame}\frametitle{Разбор летучки}

\end{frame}

\begin{frame}\frametitle{Линейная модель нейрона}
Модель МакКаллока-Питтса:\\
$f_j: X \rightarrow R, j = 1,\dots, n$ — числовые признаки\\
$a(x,w) = \sigma(\langle w, x_i \rangle) = \sigma(\sum\limits_{j=1}^n w_j f_j(x) - w_0)$\\
где $w_0, w_1, \dots,w_n \in R$ -- веса признаков\\
$\sigma(s)$ -- функция активации (например, $\sign$)

TODO: картинка
\end{frame}

\begin{frame}\frametitle{Линейные алгоритмы классификации и регрессии}
Задача классификации: \\
$Y = \left\{ \pm1 \right\}, a(x,w) = \sign \langle w, x_i \rangle $\\
$Q(w;X^l) = \sum\limits_{i=1}^l \mathcal{L} \langle w, x_i\rangle y_i \rightarrow \min\limits_w$
Задача регрессии:\\
$Y = R, a(x,w) = \sigma(\langle w, x_i \rangle)$\\
$Q(w;X^l) = \sum\limits_{i=1}^l (\sigma(\langle w, x_i \rangle) - y_i)^2 \rightarrow \min\limits_w $
\end{frame}

\begin{frame}\frametitle{Нейронная реализация логических функций}

\end{frame}


\begin{frame}\frametitle{Логическая функция XOR}

\end{frame}


\begin{frame}\frametitle{Любую ли функцию можно представить нейросетью?}
\begin{enumerate}[--]
\item Двухслойная сеть в $\left\{0, 1 \right\}^n$ позволяет реализовать произвольную булеву функцию.
\item Двухслойная сеть в $\mathbb{R}^n$ позволяет отделить произвольный выпуклый многогранник.
\item Трёхслойная сеть $\mathbb{R}^n$ позволяет отделить произвольную многогранную область, не обязательно выпуклую, и даже не обязательно связную.
\item С помощью линейных операций и одной нелинейной функции активации $\phi$ можно приблизить любую непрерывную функцию с любой желаемой точностью
\end{enumerate}
\end{frame}

\begin{frame}\frametitle{Практические рекомендации}
\begin{enumerate}[--]
\item Двух-трёх слоёв обычно достаточно.
\item Можно достраивать нейроны в произвольных местах сети по необходимости, вообще не заботясь о числе слоёв.
\end{enumerate}
\end{frame}

\begin{frame}\frametitle{Многослойная нейронная сеть}
Пусть для общности $Y = \mathbb{R}^M$, для простоты слоёв только два.\\
TODO: картинка
\end{frame}

\begin{frame}\frametitle{Стохастический градиент}
\end{frame}


\begin{frame}\frametitle{Задача дифференцирования суперпозиции функций}
\end{frame}

\begin{frame}\frametitle{Быстрое вычисление градиента}
\end{frame}


\begin{frame}\frametitle{Алгоритм обратного распространения ошибки}
\end{frame}

\begin{frame}\frametitle{Особенности}
\begin{enumerate}[+]
\item Эффективность: градиент вычисляется за время, сравнимое со временем вычисления самой сети
\item Легко обобщается на любые $\sigma$, $\mathcal{L}$
\item Возможно динамическое (потоковое) обучение
\item На сверхбольших выборках не обязательно брать все $x_i$
\item возможность распараллеливания
\end{enumerate}

\begin{enumerate}[--]
\item возможна медленная сходимость
\item застревание в локальных минимумах
\item проблема переобучения
\item подбор комплекса эвристик является искусством
\end{enumerate}
\end{frame}


\begin{frame}\frametitle{Стандартные эвристики для метода SG}
Применимы все те же эвристики, что и в обычном SG.\\

Напомните.
\end{frame}

\begin{frame}\frametitle{Стандартные эвристики для метода SG}
\begin{enumerate}[--]
\item Инициализация весов
\item Порядок предъявления объектов
\item Оптимизация величины градиентного шага
\item Регуляризация (сокращение весов)
\end{enumerate}
\end{frame}

\begin{frame}\frametitle{Новые проблемы}
\begin{enumerate}[--]
\item Выбор функций активации в каждом нейроне
\item Выбор числа слоёв и числа нейронов;
\item Выбор значимых связей;
\end{enumerate}
\end{frame}

\begin{frame}\frametitle{Ускорение сходимости}
\begin{enumerate}[--]
\item Тщательный подбор начального приближения
\item Выбивание из локальных минимумов
\item Адаптивный градиентный шаг
\item Метод сопряжённых градиентов и chunking -- разбиение суммы $Q(w) = \sum\limits_{i=1}^l \mathcal{L}_(w)$ на блоки
\end{enumerate}
\end{frame}

\begin{frame}\frametitle{Начальное приближение}
Нейроны настраиваются как отдельные линейные алгоритмы:
\begin{enumerate}[--]
\item либо по случайной подвыборке $X' \subseteq X^l$
\item либо по случайному подмножеству входов
\item либо из различных случайных начальных приближений
\end{enumerate}
Tем самым обеспечивается различность нейронов.
\end{frame}

\begin{frame}\frametitle{Улучшение сходимости}
Левенберга-Марквардта
\end{frame}

\begin{frame}\frametitle{Оптимизация структуры сети}
Динамическое наращивание сети.\\

\end{frame}

\begin{frame}\frametitle{Оптимизация структуры сети}
Прореживание сети.\\
\end{frame}


\begin{frame}\frametitle{Специальные нейронные сети}
%
\end{frame}

\begin{frame}\frametitle{На следующей лекции}
\begin{itemize}
\item[--] 
\end{itemize}
\end{frame}
\end{document}
