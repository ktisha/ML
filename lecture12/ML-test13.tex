%% -*- TeX-engine: luatex; ispell-language: russian -*-

\documentclass[a4paper,12pt]{article}

\input{handout-base}

\begin{document}
  \subsection*{Тест №12\hfill{15 мая 2015}}

  \makebox[\textwidth]{Представьтесь:\enspace\hrulefill}

  \paragraph{1} В чем суть обратного хода в алгоритме обратного распространения ошибки?

	\makebox[\linewidth]{\hrulefill}
	\makebox[\linewidth]{\hrulefill}
	\makebox[\linewidth]{\hrulefill}
	\makebox[\linewidth]{\hrulefill}
	
  \paragraph{2} Какая идея лежит в основе превращения алгоритма стохастического градиента в алгоритм обратного распространения ошибки?

	\makebox[\linewidth]{\hrulefill}
	\makebox[\linewidth]{\hrulefill}
	\makebox[\linewidth]{\hrulefill}
	\makebox[\linewidth]{\hrulefill}
	
  \paragraph{3} Как выглядит операция свертки?

  \makebox[\linewidth]{\hrulefill}
  \makebox[\linewidth]{\hrulefill}
  \makebox[\linewidth]{\hrulefill}
  \makebox[\linewidth]{\hrulefill}

  \paragraph{4} Что необходимо для приближения непрерывной функции нейронной сетью с желаемой точностью?
  
  \makebox[\linewidth]{\hrulefill}
  \makebox[\linewidth]{\hrulefill}
  \makebox[\linewidth]{\hrulefill}
  \makebox[\linewidth]{\hrulefill}  
  
  \paragraph{5} Расскажите в чем заключается суть двух способов оптимизации структуры нейросети.

  \makebox[\linewidth]{\hrulefill}
  \makebox[\linewidth]{\hrulefill}
  \makebox[\linewidth]{\hrulefill}
  \makebox[\linewidth]{\hrulefill}
\end{document}
