%% -*- TeX-engine: luatex; ispell-language: russian -*-

\documentclass[a4paper,12pt]{article}

\input{handout-base}

\begin{document}
  \subsection*{Тест №11\hfill{8 мая 2015}}

  \makebox[\textwidth]{Представьтесь:\enspace\hrulefill}

  \paragraph{1} Как решается задача регрессии для нелинейной модели?

	\makebox[\linewidth]{\hrulefill}
	\makebox[\linewidth]{\hrulefill}
	\makebox[\linewidth]{\hrulefill}
	\makebox[\linewidth]{\hrulefill}
	
  \paragraph{2} Какое неявное предположение делается о природе данных при использовании метода наименьших квадратов?

	\makebox[\linewidth]{\hrulefill}
	\makebox[\linewidth]{\hrulefill}

  \paragraph{3} Что такое проблема мультиколлинеарности и как с ней бороться?

  \makebox[\linewidth]{\hrulefill}
  \makebox[\linewidth]{\hrulefill}
  \makebox[\linewidth]{\hrulefill}
  \makebox[\linewidth]{\hrulefill}

  \paragraph{4} Для чего используется сингулярное разложение в линейной регрессии?
  
  \makebox[\linewidth]{\hrulefill}
  \makebox[\linewidth]{\hrulefill}
  \makebox[\linewidth]{\hrulefill}
  \makebox[\linewidth]{\hrulefill}  
  
  \paragraph{5} Какое требование накладывается на новые признаки в методе главных компонент?

  \makebox[\linewidth]{\hrulefill}
  \makebox[\linewidth]{\hrulefill}
  \makebox[\linewidth]{\hrulefill}
  \makebox[\linewidth]{\hrulefill}
\end{document}
