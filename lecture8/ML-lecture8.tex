\documentclass[12pt]{beamer}
\usepackage{cmap}
\usepackage[T2A]{fontenc}
\usepackage[utf8]{inputenc}
\usepackage{ifluatex}
\usefonttheme[onlymath]{serif}
\usepackage{svg}
\usepackage{enumerate}
\usepackage{hyperref}
\usepackage{mathtools}
\setbeamertemplate{footline}[frame number]
\definecolor{beamer@darkgreen}{rgb}{0,0.6,0}
\setbeamercolor{normal text}{fg=black,bg=white}
\setbeamercolor{title}{fg=black,bg=beamer@darkgreen}
\setbeamercolor{frametitle}{fg=black,bg=beamer@darkgreen}
\setbeamercolor{background canvas}{parent=normal text}

\usepackage[english,russian]{babel}
\usepackage{graphicx}
\usepackage{listings}
\DeclareMathOperator{\sign}{sign}

\author{Катя Тузова}
\title{Машинное обучение}
\date{}


\subtitle{Лекция 8. Логические алгоритмы классификации.}

\begin{document}	
\frame{\titlepage}

\begin{frame}\frametitle{Разбор летучки}

\end{frame}


\begin{frame}\frametitle{Логические закономерности}
${X^l = \left( x_i, y_i \right)_{i=1}^l}$ - обучающая выборка.\\
Логическая закономерность (правило) -- предикат ${R: X \rightarrow \left\{ 0, 1 \right\} }$, который удовлетворяет двум требованиям:\\
\begin{enumerate}
	\item Интерпретируемость
	\item Информативность относительно одного из классов ${c \in Y}$
\end{enumerate}
\end{frame}


\begin{frame}\frametitle{Интерпретируемость}
	\begin{itemize}
		\item[--] Записывается на естественном языке
		\item[--] Зависит от небольшого числа признаков
	\end{itemize}
\end{frame}

\begin{frame}\frametitle{Информативность}
	\begin{itemize}
		\item[--] ${ p_c(R) = \# \left\{ x_i: R(x_i) = 1 , y_i = c \rightarrow \max \right\} }$
		\item[--] ${ n_c(R) = \# \left\{ x_i: R(x_i) = 1 , y_i \neq c \rightarrow \min \right\} }$
	\end{itemize}
	\textcolor{red}{TODO: example}
\end{frame}

\begin{frame}\frametitle{Основные проблемы}
	\begin{itemize}
		\item[--] Как изобретать признаки? 
		\item[--] Какого вида закономерности $R$ нужны?
		\item[--] Как определить информативность		
		\item[--] Как искать закономерности		
		\item[--] \textcolor{red}{TODO}
	\end{itemize}
\end{frame}

\begin{frame}\frametitle{Оценивание информативности}
	Как свернуть 2 критерия в один?\\
	Очевидные свертки:\\
	\begin{itemize}
		\item[--] \textcolor{red}{TODO}
		\item[--] 
	\end{itemize}
\end{frame}

\begin{frame}\frametitle{Пример свертки двух критериев}
\textcolor{red}{TODO: picture}
\end{frame}

\begin{frame}\frametitle{Используемые критерии информативности}
\textcolor{red}{TODO}
\end{frame}

\begin{frame}\frametitle{Энтропийный критерий информативности}
\textcolor{red}{TODO}
\end{frame}


\begin{frame}\frametitle{Тест Фишера}
\textcolor{red}{TODO}
\end{frame}

\begin{frame}\frametitle{Критерий Джини}
\textcolor{red}{TODO}
\end{frame}

\begin{frame}\frametitle{Виды правил}
\begin{itemize}
\item[--] Пороговое условие
\item[--] Конъюнкция пороговых условий
\item[--] Синдром -- выполнение не менее d условий из J
\item[--] Полуплоскость — линейная пороговая функция
\item[--] Шар -- пороговая функция близости
\end{itemize}
\textcolor{red}{TODO}
\end{frame}

\begin{frame}\frametitle{Поиск информативных закономерностей}
\textcolor{red}{TODO}
\end{frame}

\begin{frame}\frametitle{Критерий Парето}
\textcolor{red}{TODO}
\end{frame}

\begin{frame}\frametitle{Жадный алгоритм построения решающего списка}
\textcolor{red}{TODO}
\end{frame}

\begin{frame}\frametitle{Недостатки}
\textcolor{red}{TODO}
\end{frame}

\begin{frame}\frametitle{Бинарное решающее дерево}
\textcolor{red}{TODO}
\end{frame}

\begin{frame}\frametitle{Пример}
\textcolor{red}{TODO}
\end{frame}

\begin{frame}\frametitle{Алгоритм построения ID3}
\textcolor{red}{TODO}
\end{frame}

\begin{frame}\frametitle{Достоинства и недостатки}
\textcolor{red}{TODO}
\end{frame}

\begin{frame}\frametitle{Усечение дерева C4.5}
\textcolor{red}{TODO}
\end{frame}

\begin{frame}\frametitle{ODT}
\textcolor{red}{TODO}
\end{frame}

\begin{frame}\frametitle{Random forest}
\textcolor{red}{TODO}
\end{frame}

\begin{frame}\frametitle{На следующей лекции}
\begin{itemize}
\item[--] Байесовские методы классификации
\end{itemize}
\end{frame}
\end{document}
