\documentclass[12pt]{beamer}
\usepackage{cmap}

\usepackage{fontspec}
\setmainfont{PT Serif}
\newfontfamily\cyrillicfont[Script=Cyrillic,Ligatures=TeX]{PT Serif}
\setsansfont{PT Sans}
\setmonofont[Ligatures=NoCommon]{PT Mono}
\defaultfontfeatures{Ligatures=TeX}
\usepackage[bold-style=ISO]{unicode-math}
\setmathfont{XITS Math}
\usepackage{ifluatex}
\usefonttheme[onlymath]{serif}
\usepackage{svg}
\usepackage{enumerate}
\usepackage{hyperref}
\usepackage{mathtools}
\setbeamertemplate{footline}[frame number]
\definecolor{beamer@darkgreen}{rgb}{0,0.6,0}
\setbeamercolor{normal text}{fg=black,bg=white}
\setbeamercolor{title}{fg=black,bg=beamer@darkgreen}
\setbeamercolor{frametitle}{fg=black,bg=beamer@darkgreen}
\setbeamercolor{background canvas}{parent=normal text}

\usepackage[english,russian]{babel}
\usepackage{graphicx}
\usepackage{listings}

\author{Катя Тузова}
\title{Машинное обучение}
\subtitle{Лекция 6. Метод опорных векторов.}
\date{}

\begin{document}
\frame{\titlepage}

\begin{frame}\frametitle{Разбор летучки}
\end{frame}

%\begin{frame}\frametitle{Максимизиция отступа}
%Идея:\\
%Максимизировать отступ между
%двумя параллельными опорными плоскостями, а затем
%провести параллельную им плоскость на равных расстояниях.\\
%\vspace{5mm}
%Как посчитать расстояние от точки до гиперплоскости?
%\end{frame}

%\begin{frame}\frametitle{Постановка задачи}
%Задача классификации:
%$X = \mathbb{R}^n$, ${Y = \left\{ -1, +1\right\}}$\\
%${X^l = (x_i, y_i)_{i = 1}^l}$ -- обучающая выборка\\
%\vspace{5mm}
%Найти:\\
%$w \in \mathbb{R}^n, w_0 \in \mathbb{R}$ -- параметры алгоритма классификации\\
%${a(x, w) = sign(\langle x, w\rangle + w_0) }$
%\end{frame}

\begin{frame}\frametitle{}
\end{frame}

\end{document}
